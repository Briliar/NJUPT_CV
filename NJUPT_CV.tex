\iffalse
	***************************************************************************************
	* 这是一个NJUPT 简历模板 (Fork 自网络开源项目如https://github.com/LeyuDame/BNUCV/tree/main)
	* * [核心警告]:
	* 必须使用 XeLaTeX 编译链!必须编译三次!
	* 第一次编译是为了告诉 LaTeX 它是个文档;
	* 第二次编译是为了让图片知道它该在哪;
	* 第三次编译是为了祈祷排版不炸。
	* * [VS Code 配置玄学]:
	* 如果你用 VS Code + LaTeX Workshop,请确保 "recipes" 里配置了 "xelatex -> xelatex -> xelatex"
	* 问就是玄学,少一次都不行。
	*
	* [食用指南]:
	* 1. 章节顺序随便改:觉得项目太水就把“技能”往前挪,觉得技能太菜就把“奖项”往前挪。
	* 2. 这里的宏包就像堆积木,抽掉一根可能整个文档就崩了,非必要别动 \usepackage。
	* 3. 祝大家都有 Offer,只要胆子大,简历写成架构师。
	***************************************************************************************
\fi

\documentclass[11pt]{article}

% ----------- 宏包加载区 (没事别动,动了报错别找我) -----------
\usepackage{xltxtra}
\usepackage{bookmark}
\usepackage{hyperref}
\hypersetup{hidelinks} % 隐藏超链接那个丑陋的方框,假装是纯文本
\usepackage{url}
\urlstyle{tt}
\usepackage{multicol}
\usepackage{xcolor}
\usepackage{calc}
\usepackage{graphicx}
\usepackage{tikz}
\usetikzlibrary{calc}
\usepackage{fontspec}
\usepackage{xeCJK}
\usepackage{relsize}
\usepackage{xspace}
\usepackage{fontawesome} % 各种图标全靠它,没有它简历少一半逼格
\usepackage{titlesec}
\usepackage{enumitem}
\usepackage{siunitx}
\usepackage{amssymb}
\usepackage{tabularx}

% ----------- 各种自定义黑魔法 -----------

% 取消中英文字符间距 (强迫症福音)
\CJKsetecglue{}							            

% 这是一个看起来很高级的 C++ 写法。
% 直接写 C++ 会显得很土,用这个命令会显得你很懂底层 (虽然其实只是调整了加号的位置)。
\protected\def\Cpp{{C\nolinebreak[4]\hspace{-.05em}\raisebox{.28ex}{\relsize{-1}++}}\xspace}

% 全局取消缩进,写代码的都不喜欢首行缩进
\setlength{\parindent}{0pt}							

% 取消页码,简历就一页纸,要啥自行车
\pagenumbering{gobble}								

% 列表间距调整 (为了在有限的A4纸里塞下更多吹牛的内容)
\setlist[itemize]{topsep=0em, leftmargin=*}		
\setlist[enumerate]{topsep=0em, leftmargin=*}	

% --- 标题格式设置 ---
\titleformat{\section}					    
  {\LARGE\bfseries\raggedright} 		      
  {}{0em}                      			  
  {}                           			  
  [{\color{Blue}\titlerule}]           
\titlespacing*{\section}{0cm}{*1.2}{*1.2}	

\titleformat{\subsection}				    
  {\large\bfseries\raggedright} 		      
  {}{0em}                      			  
  {}                           			  
  []
\titlespacing*{\subsection}{0cm}{*1.2}{*1.2}

% --- 页面几何学 ---
% 这里的边距经过了精密的计算 (指瞎调的),能刚好塞满一页看起来很充实
\usepackage[
	a4paper,
	left=1.2cm,
	right=1.2cm,
	top=1.5cm,
	bottom=1cm,
	nohead
]{geometry}

% 中文字符间距
\renewcommand{\CJKglue}{\hskip 0.05em}

% --- 字体设置  ---
\setmainfont[
    Path=fonts/,
    Extension=.ttf,
    BoldFont=* Bold,
]{Microsoft Yahei}

\setCJKmainfont[
    Path=fonts/,
    Extension=.ttf,
    BoldFont=* Bold,
]{Microsoft Yahei}

% 主题色:南邮蓝 (大概是这个色吧,不是也得是)
\definecolor{Blue}{RGB}{0, 80, 158}

% 行间距:调大一点显得字多,调小一点显得内容多,目前是 1.2 倍
\renewcommand{\arraystretch}{1.2}
\linespread{1.2}

%%%%%%%%%%%%%%%%%%%%%%%%%%%%%%%%%%%%%%%%%%%%%%%%%%%%%%%%%%%%%%%%%%%%%%%%%%%%%%%%%%%%%%%%%%%%%%%%%%%%%%%%%%%%%%%%%%%%%%%%%%%%%%%%%%%%%%%%%%%%%%%%%%
%    !!!!!!!! 变量定义区 !!!!!!!!
%%%%%%%%%%%%%%%%%%%%%%%%%%%%%%%%%%%%%%%%%%%%%%%%%%%%%%%%%%%%%%%%%%%%%%%%%%%%%%%%%%%%%%%%%%%%%%%%%%%%%%%%%%%%%%%%%%%%%%%%%%%%%%%%%%%%%%%%%%%%%%%%%%
% 学院
\newcommand{\school}{自动化学院 | College of Automation}

% 联系方式
\newcommand{\contact}
{
    \small              
    \scriptsize         
    \textcolor{white}
    {
        % 邮箱:建议别用QQ邮箱,显得像小学生 (虽然我们内心就是)
        \faEnvelope \quad \href{mailto:xxxx@njupt.edu.cn}{xxxx@njupt.edu.cn}    
        \hspace{6em}   
        % 微信:方便HR加你好友发好人卡 
        \faWechat \quad wx\_id\_placeholder    
        \hspace{6em}    
        \faPhone \quad 1XX-XXXX-XXXX                  
        %\hspace{4em}   
        % GitHub:没有绿格子的就别放了,丢人 
        %\faGithub \quad \href{https://github.com/xxxx}{https://github.com/xxxx}         
    }
}

%%%%%%%%%%%%%%%%%%%%%%%%%%%%%%%%%%%%%%%%%%%%%%%%%%%%%%%%%%%%%%%%%%%%%%%%%%%%%%%%%%%%%%%%%%%%%%%%%%%%%%%%%%%%%%%%%%%%%%%%%%%%%%%%%%%%%%%%%%%%%%%%%%
%    !!!!!!!! 正文开始:请开始你的表演 !!!!!!!!
%%%%%%%%%%%%%%%%%%%%%%%%%%%%%%%%%%%%%%%%%%%%%%%%%%%%%%%%%%%%%%%%%%%%%%%%%%%%%%%%%%%%%%%%%%%%%%%%%%%%%%%%%%%%%%%%%%%%%%%%%%%%%%%%%%%%%%%%%%%%%%%%%%
\begin{document}
	
	% --- 这一坨是页眉,动了这代码,校徽可能会飞到月球上去 ---
	\begin{tikzpicture}[remember picture, overlay]
		\node[anchor=north, inner sep=0pt](header) at (current page.north){
			\includegraphics[width=\paperwidth]{images/header.png}
		};
		\node[anchor=west](school_logo) at (header.west){
			\hspace{0.5cm}
			\includegraphics[width=0.25\textwidth]{images/njupt_logo.png}
		};
		\node[anchor=east](school_name) at(header.east){
			\textcolor{white}{\textbf{\school}}
			\hspace{0.5cm}
		};
	\end{tikzpicture}
	\vspace{-4em}

	% --- 这一坨是页脚 ---
	\begin{tikzpicture}[remember picture, overlay]
		\node[anchor=south, inner sep=0pt](footer) at (current page.south){
			\includegraphics[width=\paperwidth]{images/footer.png}
		};
        % 联系方式
        \node[anchor=center] at(footer.center){\contact};
	\end{tikzpicture}
	
	% --- 这一坨是背景大Logo,透明度调得很低,防止喧宾夺主 ---
	\begin{tikzpicture}[remember picture, overlay]
		\node[opacity=0.1] at(current page.center){
			\includegraphics[width=0.7\paperwidth, keepaspectratio]{images/njupt.png}
		};
	\end{tikzpicture}

	% --- 个人信息区域 ---
    \begin{figure}[h]
        % 左半边:文字信息
        \begin{minipage}{0.87\textwidth}
            \section{\makebox[\widthof{\faUser}][c]{\color{Blue}{\faUser}}\quad 个人信息}
            \begin{tabularx}{\linewidth}{p{\widthof{出生日期:}}Xp{\widthof{政治面貌:}}X}
                姓名: & ikun & 性别: & 头发尚在 \\
                出生日期: & 200X年XX月 & 政治面貌: & 群众/党员 
                % 别写太多,写多了HR也记不住
            \end{tabularx}
        \end{minipage}
        % 右半边:证件照
        % 建议P得稍微帅一点,但别P到亲妈都不认识,面试会尴尬
        \begin{minipage}{0.12\textwidth}
            \includegraphics[width=\linewidth]{images/ikun.jpg}
        \end{minipage}
    \end{figure}

    % --- 教育背景 ---
    \vspace{-2em}
	\section{\makebox[\widthof{\faGraduationCap}][c]{\color{Blue}{\faGraduationCap}}\quad 教育背景}
	\vspace{-1em}
    \begin{table}[h!]
        \begin{tabularx}{\textwidth}{XXp{\widthof{2021年 -- 2025年}}}
            南京邮电大学 & 电子信息 (硕士在读) & 2024年 -- 至今\\
            南京邮电大学 & 自动化 (本科) & 2020年 -- 2024年\\
            % \mbox{主修课程:} 写两个分高的就行了,别把挂科的写上去
        \end{tabularx}
    \end{table}

    % --- 技能特长 (整活重点区域) ---
    % 建议:把会的写在前面,“了解”的写在最后(防露馅)
    % \faGears 这是齿轮,适合机械类,我电信的也喜欢齿轮,就用这个了
    % \faFlask 这是烧瓶,适合生化类
    % \faLaptop 这是个笔记本电脑,适合计算机类
    % \faUsers 这是三个人,适合商科
    \vspace{-1em}
    \section{\makebox[\widthof{\faWrench}][c]{\color{Blue}{\faWrench}}\quad 技能特长}
    \vspace{0.5em}
    \begin{itemize}
        \item 熟练掌握 \Cpp 、Python 等语言的 \textbf{Hello World} 写法。
        \item 熟悉 Linux 系统常用命令,\textbf{精通 rm -rf /* 与系统重装流程}。
        \item 熟练掌握 Git 版本控制工具,\textbf{擅长使用 git push -f --no-verify 覆盖队友代码}。
        \item 熟练使用 \textbf{Ctrl+C / Ctrl+V} 进行面向搜索引擎编程 (Stack Overflow 资深复制员)。
        \item 了解 深度学习 (指会调包)、强化学习 (指跑通 Demo)、知识蒸馏 (指听说过名字)。
        \item 熟练使用 Qt Creator 进行界面开发 (指只会拖控件)。
        \item 具备良好的英语阅读能力 (指熟练使用 Google Translate 阅读报错信息)。
    \end{itemize}

    % --- 项目经历 (学术包装艺术) ---
    % 小技巧:
    % "做了什么" -> "设计了基于xx协议的高可用架构"
    % "修了Bug" -> "攻克了复杂场景下的鲁棒性难题"
    \section{\makebox[\widthof{\faGears}][c]{\color{Blue}{\faGears}}\quad 项目经历}
    \vspace{0.5em}
    
    % --- 项目 1 ---
    \subsection{基于 Docker 的云端高并发算命系统 \hfill 个人全栈项目}
    
    \textbf{第一负责人} \hfill 2024年6月 - 至今
    
    \textbf{(GPT辅助生成的描述)} 针对传统算命行业响应慢、无法跨平台的问题,设计了一套基于 \textbf{Spring Boot + Vue3} 的分布式算命平台。
    
    \begin{itemize}
        \item 引入 \textbf{Redis} 缓存八字运势数据,将算命接口 QPS 提升至 10w+。
        \item 核心算法采用“伪随机数生成”策略,完美模拟了天机不可泄露的玄学特征。
        \item 成功将应用容器化部署于 256M 内存的学生机上,并设计了 24 小时自动重启的“防宕机”脚本。
    \end{itemize}
    
    \vspace{1em} 

    % --- 项目 2 ---
    \subsection{基于多模态大模型的导师情绪监测与预警系统 \hfill 实验室划水项目}

    \textbf{核心搬砖工} \hfill 2023年9月 - 2024年1月
    
    \textbf{(GPT辅助生成的描述)} 针对研究生实验室生存环境恶劣的问题,提出了一种基于计算机视觉的情绪分析算法。
    
    \begin{itemize}
        \item 利用 \textbf{YOLOv8} 检测导师是否出现在实验室门口,检测精度达到 99.9\%。
        \item 结合音频信号分析导师脚步声频率,实现了提前 30 秒的“摸鱼-工作”状态自动切换功能。
        \item 该系统有效降低了组会的被骂率,提升了实验室整体幸福感指数。
    \end{itemize}
    
    
    \newpage
     % 终于用完一页了,加一页展示怎么加页眉页脚
     % --- 这一坨是页眉,动了这代码,校徽可能会飞到月球上去 ---
    \begin{tikzpicture}[remember picture, overlay]
		\node[anchor=north, inner sep=0pt](header) at (current page.north){
			\includegraphics[width=\paperwidth]{images/header.png}
		};
		\node[anchor=west](school_logo) at (header.west){
			\hspace{0.5cm}
			\includegraphics[width=0.25\textwidth]{images/njupt_logo.png}
		};
		\node[anchor=east](school_name) at(header.east){
			\textcolor{white}{\textbf{\school}}
			\hspace{0.5cm}
		};
	\end{tikzpicture}
	\vspace{-4em}

	% --- 这一坨是页脚 ---
	\begin{tikzpicture}[remember picture, overlay]
		\node[anchor=south, inner sep=0pt](footer) at (current page.south){
			\includegraphics[width=\paperwidth]{images/footer.png}
		};
        % 联系方式
        \node[anchor=center] at(footer.center){\contact};
	\end{tikzpicture}
	
	% --- 这一坨是背景大Logo,透明度调得很低,防止喧宾夺主 ---
	\begin{tikzpicture}[remember picture, overlay]
		\node[opacity=0.1] at(current page.center){
			\includegraphics[width=0.7\paperwidth, keepaspectratio]{images/njupt.png}
		};
	\end{tikzpicture}
    
    % --- 竞赛经历 ---
    \section{\makebox[\widthof{\faTrophy}][c]{\color{Blue}{\faTrophy}}\quad 竞赛/科研经历}
    \vspace{-1em}
    \begin{table}[h!]
        \begin{tabularx}{\textwidth}{Xp{\widthof{第零负责人}}p{\widthof{国家级-第100名}}p{\widthof{2030年13月}}}
            \textbf{2024年“互联网+”熬夜大赛} & 个人参赛 & 没得奖 & 2024年11月 \\
            \textbf{某 SCI 二区论文 (在投)} & 既然是二区 & 当然是一作 & 审稿中...\\
            \textbf{2023年全国大学生退堂鼓演奏大赛} & 首席鼓手 & 省级-特等奖 & 2023年12月\\
        \end{tabularx}
    \end{table}

    % --- 其他 ---
    \section{\makebox[\widthof{\faInfo}][c]{\color{Blue}{\faInfo}}\quad 其他}
    \begin{itemize}
        \item 英语水平:CET-6 (能流畅阅读英文游戏菜单)
        \item 拥有极强的抗压能力 (指连续熬夜三天改论文,且发际线仅后移 1cm)
        \item 具有团队协作精神 (指除了写代码什么都干,包括买饭、拿快递、给服务器除尘)
        \item 文字排版:熟练使用 \LaTeX \ (指会套用模板并解决报错)
    \end{itemize}

\end{document} 